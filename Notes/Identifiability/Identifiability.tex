\documentclass[12pt,a4paper]{article}
\usepackage[truedimen,margin=30mm]{geometry} 

%

%=usepackage================================================
\usepackage{epsfig,hyperref,url}
\usepackage{amsmath}
\usepackage{amsfonts}
\usepackage{amsthm}
\usepackage{amssymb}
\usepackage{ascmac}
\usepackage{booktabs}
\usepackage{color}
\usepackage{graphicx}
\usepackage{bm}
\usepackage{multirow}
\usepackage{setspace}
\usepackage{natbib}
\usepackage{mathtools}
\mathtoolsset{showonlyrefs}
\usepackage{color}
\usepackage{csquotes}
\usepackage{enumitem}
\setlist[enumerate]{wide=0pt, leftmargin=15pt, labelwidth=15pt, align=left}
\usepackage{mathrsfs}
\usepackage{comment}
\usepackage{times}
\usepackage{here}
\usepackage{listings}
\lstset{
  basicstyle={\ttfamily},%
  identifierstyle={\small},%
  commentstyle={\smallitshape},%
  keywordstyle={\small\bfseries},%
  ndkeywordstyle={\small},%
  stringstyle={\small\ttfamily},%
  frame={tb},%
  breaklines=true,%
  columns=[l]{fullflexible},%
  numbers=left,%
  xrightmargin=0zw,%
  xleftmargin=2zw,%
  numberstyle={\scriptsize},%
  stepnumber=1,%
  numbersep=1zw,%
  lineskip=-0.5ex,%
  backgroundcolor={\color[gray]{0.95}},%
}


%
\usepackage{titlesec}
\titleformat*{\section}{\large\bfseries}
\titleformat*{\subsection}{\it}


%
\newtheorem{df}{Definition}
\newtheorem{thm}{Theorem}
\newtheorem{lem}{Lemma}
\newtheorem{cor}{Corollary}
\newtheorem{prp}{Proposition}
\newtheorem{exm}{Example}
\newtheorem{algo}{Algorithm}
\newtheorem{as}{Assumption}
%




%----------------------------------------%
%             Definition                 %
%----------------------------------------%

\def\ep{{\varepsilon}}
\def\Si{{\Sigma}}
\def\Ga{\Gamma}

\def\tht{{\widetilde{\theta}}}
\def\thh{{\widehat{\theta}}}



%----------------------------------------%
%           Title page                   %
%----------------------------------------%
\title{{\bf Identifiability of the Trans-dimensional Bradley-Terry Model}\footnote{\today}}
\author{\empty}
\date{\empty}


\begin{document}

\maketitle
\doublespacing



%----------------------------------------%
%            Introduction                %
%----------------------------------------%
\section{Identification of the Likelihood}
Consider the sampling distribution for a response $\pi$ in a trans-dimensional Bradley-Terry (TDBT) model given by
\begin{equation} \label{eq:model}
  p(Y \mid w, F) = \prod_{i<j} \frac{\exp\{w^\top (f_i-f_j)\}}{1+\exp\{w^\top (f_i-f_j)\}}.
\end{equation}
where $Y = \{ y_{12}, \dots, y_{N-1, N} \}$, $w = (w_{1}, \dots, w_{N})^{\top}$ and $F = \{ f_{1}, \dots, f_{N} \}$.
When $K=1$ and $w=1$, the model \eqref{eq:model} reduces to the conventional Bradley-Terry model \citep{bradley1952Rank}.
The multiple unobservable parameters in the equation \eqref{eq:model} induce the problem of identification in the general data model $\prod_{i<j} \pi_{ij}$.

Identification can be dealt with by specifying additional assumptions and constraints that render the model estimable.
In general, we say that a statistical model is identifiable if distinct parameter values generate different joint probability distributions.
The likelihood of a TDBT model can be underidentified for the following five reasons: location invariance, scale invariance, reflective invariance, label switching, and rotation invariance.


 
\section*{Location and Scale Invariance}
For illustration, consider the TDBT model with the linear predictor $z_{ij} = w (f_{i} - f_{j})$.
If we add a constant vector $\delta = (\delta_{1}, \dots, \delta_{K})^{\top}$ to both $f_{i}$ and $f_{j}$, the likelihood still remains the same.
In addition, if we divide $w$ by a constant vector $c = (c_{1}, \dots, c_{K})^{\top}$ ($c_{k} \in \mathbb{R} \backslash \{0\}$) and multiply both $f_{i}$ and $f_{j}$ by the same constant, the likelihood remains unchanged:
\begin{equation}
    p(Y \mid w, F) = p \Bigl(Y \mid \tilde w, \tilde F \Bigr)
\end{equation}
where $\tilde w = \bigl(\tfrac{w_1}{c_1}, \dots, \tfrac{w_K}{c_K} \bigr)^{\top}$, $\tilde F = \{ \tilde f_{1}, \dots, \tilde f_{N} \}$ and $\tilde f_{n} = (c_{1} (f_{n1} + \delta_{1}), \dots, c_{K} (f_{nK} + \delta_{K}))^{\top}$ for $n=1,\dots,N$; $k=1,\dots,K$.
To address this issue, one method is to assign a very strong prior to $f_{i}$ \citep[e.g.,][]{bafumi2005Practical}, a fixed population distribution that can determine the location and scale of the parameter space for the entire likelihood function $p(Y \mid w,F)$.
The constraint is simply
\begin{equation}
    f_{i} \sim \mathcal{N}_{K} (\mathbf{0}, \boldsymbol{I}) \quad \text{for} \ i= 1, 2, \dots, N.
\end{equation}





\section*{Reflective Invariance}
Even after eliminating additive and multiplicative aliasing, the likelihood of the TDBT model remains invariant when every element of the $k$-th dimension is multiplied by $-1$ simultaneously in the weight vector $w$ and in all worth vectors $f_{i}$, so there exists $2^{K}$ equivalent parameterizations.
A common remedy to remove this ambiguity is to impose a sign-identification rule, for example, to restrict the support of $w$ to the positive orthant as follows:
\begin{equation}
    w_{k}>0 \quad\text{for}\;k=1,\dots,K.
\end{equation}
either by imposing $w$ a truncated prior or by re-parameterising $w$ (e.g., $\log w_{k} \in \mathbb{R}$).
Since the sign of each element of $w$ can no longer change, the sign of $f_{i}$ is identifiable and still capture both positive and negative latent traits. 
Therefore, this positivity constraint resolves reflective invariance without affecting model flexibility.




\section*{Label Switching}
In multidimensional setting, another issue arises.
For $K$ dimensions, there are $K!$ permutations of parameter labels that yield the same likelihood. 
Combined with sign reversals, the likelihood has $2^{K} \times K!$ symmetric modes.




\section*{Rotation Invariance}





%   Reference
\bibliographystyle{chicago}
\bibliography{2025_TDBT}



\end{document}
